\documentclass{article}

\usepackage{graphicx}
\usepackage{xcolor}

% For rounded framed boxes to highlight citations
\usepackage[framemethod=tikz]{mdframed}

% Lorem Ipsum Parargraph
\usepackage{lipsum}

\usepackage[english]{babel}
\usepackage{csquotes}
\usepackage[
  backend=biber,
  style=ieee,
  sorting=none,
  maxcitenames=1,     % In-text: 1 author + et al.
  mincitenames=1,
  maxbibnames=1,      % Bibliography: 1 author + et al.
  minbibnames=1
]{biblatex}
\addbibresource{references.bib}


\begin{document}

\noindent{\LARGE\textbf{\color{blue}Annotated Bibliography II}} \hfill Trison Li\\
\noindent AP Research - Great Mills High School \hfill October 17, 2025

\vspace{6pt}
\noindent\rule{\textwidth}{2pt}

\tableofcontents
\thispagestyle{empty}
\newpage
\section{Nomenclature}

\begin{longtable}{p{0.12\textwidth} p{0.35\textwidth} p{0.45\textwidth}}
\toprule
\textbf{Acronym} & \textbf{Definition} & \textbf{Context / Related Terminology} \\
\midrule
\endfirsthead

\multicolumn{3}{c}{\textit{Nomenclature and Acronyms (continued)}} \\
\toprule
\textbf{Acronym} & \textbf{Definition} & \textbf{Context / Related Terminology} \\
\midrule
\endhead

\bottomrule
\endfoot

\bottomrule
\endlastfoot

AP & Ammonium Perchlorate ($\text{NH}_4\text{ClO}_4$) & Most common oxidizer in composite solid propellants (CSPs). \\
HTPB & Hydroxyl-Terminated Polybutadiene & A widely used polymeric binder and fuel in composite propellants. \\
CSPs & Composite Solid Propellants & Heterogeneous mixtures of oxidizer, fuel, and binder (e.g., AP/HTPB). \\
$I_{sp}$ & Specific Impulse & A measure of propellant performance efficiency (usually in seconds). \\
$\Delta H^\circ_{comb}$ & Heat of Combustion & Energy released during combustion (enthalpy change). \\
TG-DSC & Thermogravimetry-Differential Scanning Calorimetry & Combined thermal analysis technique used to study decomposition. \\
LTD & Low-Temperature Decomposition & The first irreversible decomposition step of AP, typically below $\sim 300^\circ\text{C}$. \\
HTD & High-Temperature Decomposition & The second decomposition step of AP, typically occurring at higher temperatures and involving sublimation. \\
$T_p$ & Peak Temperature & The temperature corresponding to the maximum heat flow or reaction rate during thermal analysis (DSC/DTA). \\
$E_a$ & Activation Energy & The minimum energy required to initiate a chemical reaction (often calculated using isoconversional methods). \\
KAS & Kissinger-Akahira-Sunose & An isoconversional kinetic method for calculating activation energy. \\
FWO & Flynn-Wall-Ozawa & An isoconversional kinetic method for calculating activation energy. \\
XRD & X-Ray Diffraction & Analytical technique used to determine the crystalline structure of materials. \\
SEM & Scanning Electron Microscopy & Analytical technique used to visualize surface morphology and microstructure. \\
EDS/EDAX & Energy Dispersive X-ray Spectroscopy / Analysis & Used with SEM to determine the elemental composition of materials. \\
FTIR & Fourier-Transform Infrared Spectroscopy & Analytical technique used to identify chemical functional groups and evolved gaseous products. \\
MS & Mass Spectrometry & Analytical technique used to identify evolved gaseous products based on mass-to-charge ratio. \\
TG-MS & Thermogravimetry-Mass Spectrometry & Combined thermal analysis technique for detecting decomposition products. \\
CC & Copper Chromite ($\text{CuCr}_2\text{O}_4$) & A popular catalyst/burning rate modifier for AP-based propellants. \\
MOFs & Metal-Organic Frameworks & Class of materials used as catalysts (e.g., Fe-based MOFs). \\
TMOs & Transition Metal Oxides & A class of metal compounds commonly used as catalysts. \\
NPs/NMs & Nanoparticles / Nanomaterials & Materials with components in the nanometer range, often exhibiting superior catalytic activity. \\
$\mu$Al & Micron Aluminum & Aluminum particles in the micrometer size range. \\
CTPB & Carboxyl Terminated Polybutadiene & A type of polymeric binder. \\
PBAN & Polybutadiene Acrylonitrile & A type of polymeric binder. \\
TDI & Toluene Diisocyanate & A type of curing agent used for HTPB binders. \\
IPDI & Isophorone Diisocyanate & A type of curing agent used for HTPB binders. \\
DDI & Diphenyl Diisocyanate & A type of curing agent. \\
NASA-CEA2 & NASA Chemical Equilibrium with Applications (version 2) & Computational thermochemical tool based on Gibbs free-energy minimization. \\
PROPEP 3.0 & Propellant Evaluation Program & Computational thermochemical software for assessing theoretical performance. \\
RPA & Rocket Propulsion Analysis & Computational tool for thermochemical analysis focusing on nozzle design/performance. \\
CPROPEP & CPROPEP & GNU-Octave script tool for equilibrium calculations based on NASA-CEA algorithms. \\
GUIPEP & GUIPEP & Computational tool for thermochemical equilibrium prediction. \\
LAMMPS & Large-Scale Atomic/Molecular Massively Parallel Simulator & Software used for atomic-scale molecular dynamics (MD) simulations. \\
ReaxFF & Reactive Force Field & Potential function used in MD simulations to model chemical reactions. \\
MD & Molecular Dynamics & Computational method for simulating the physical movements of atoms and molecules. \\
NNP & Neural Network Potential & A type of potential model derived from ab initio calculations for MD simulations. \\
NEPE & Nitrate Ester Plasticized Polyether & A type of high-performing propellant formulation (e.g., NEPE-75). \\
CCPs & Condensed Combustion Products & Solid residues formed during propellant combustion, often containing aluminum oxide ($\text{Al}_2\text{O}_3$). \\
NC & Nitrocellulose & A binder and energetic component used in homogeneous and double-base propellants. \\
NG & Nitroglycerin & An energetic plasticizer used in double-base propellants. \\
DB & Double Base & Propellants characterized by an energetic matrix (e.g., NC and NG). \\
CMDB & Composite Modified Double Base & Homogeneous propellants containing heterogeneous additives (like AP or Al). \\
DSC & Differential Scanning Calorimetry & Thermal analysis technique measuring heat flow associated with transitions and reactions. \\
TGA & Thermogravimetric Analysis & Thermal analysis technique measuring weight change as a function of temperature. \\
RMSE & Root Mean Square Error & A measure of prediction accuracy. \\
MAPE & Mean Absolute Percentage Error & A measure of prediction accuracy. \\
$R^2$ & Correlation Coefficient (Coefficient of Determination) & A statistical measure of how well a regression line represents data (often used for accuracy). \\
VST & Vacuum Stability Test & A compatibility test to measure gas release. \\

\end{longtable}


\newpage
\section{Questions}
1. Primary Question: How accurate are computational thermochemical tools in predicting experimentally measured metrics across AP-based formulations containing various catalytic metal additives?

2. Secondary Question 1: Do predictive errors correlate with specific catalyst types, concentrations, or other formulation discrepancies?

3. Secondary Question 2: Is atomic-scale molecular dynamics simulations (LAMMPS w/ ReaxFF) detailed enough to reveal mechanisms that can explain the variance between equilibrium-based thermochemical predictions and empirical observation within catalyzed systems?

\newpage
\section{Sources}

These sources and their analysis only include the ones considered "informational", as they provide a basis for understanding the empirical effects of catalysts and formulation parameters on AP decomposition and propellant performance. 

Sources that primarily focus on computational methods without empirical data (e.g., purely theoretical studies or software manuals) are excluded.




\newpage
\section{Data Tables}

\begin{center}
\resizebox{\textwidth}{!}{\input{table_output.tex}}
\captionof{table}{Sources and their classification}
\label{tab:empirical}
\end{center}

\newpage
\section{Overall Notes}

\subsection{Computational Tool Accuracy and Limitations}

\begin{itemize}
    \item Thermodynamic Limits - Theoretical performance parameters, such as specific impulse ($I_{sp}$) and optimal ingredient ratios, are calculated using equilibrium-based software like PROPEP 3.0 and NASA-CEA2 \cite{Azizi2024} \cite{Boukeciat2025}.
    \item Known Limitations
    \begin{enumerate}
        \item Equilibrium Models Often Overpredict Performance: These tools rely on thermodynamic equilibrium, which assumes complete reaction and ignores kinetic barriers \cite{Wang2025}. This leads to discrepancies, such as predicting a high theoretical adiabatic flame temperature ($T_{ad}$) for a formulation that fails to combust vigorously in reality, highlighting a critical limitation when kinetics are dominant (Wang, 2025).
        \item Reliance on Equilibrium: Computational tools such as NASA Chemical Equilibrium with Applications (CEA2) and PROPEP 3.0 operate on the assumption of equilibrium thermochemistry, meaning they predict propellant performance ($I_{sp}$) based on the assumption that all combustion products reach an optimized, stable chemical state \cite{Wang2025} \cite{Song2008}.
        \item Limited Validation: Despite these computer tools existing and being developed since the late 20th century, a thorough, systemic validation against a diverse range of experimentally measured propellant formulations (those containing various catalysts) remains limited.
        \item Current Status: Without quantitative documentation, the results from these digital prediction tools are generally treated only as rough estimates. The lack of quantifiable accuracy using statistical metrics (like $R^2$ or MAPE) makes researchers hesitant to rely on computational screening as a substitute for experimental validation.
    \end{enumerate}
    \item Validation Requirements - Empirical Data are Needed for Validation: The calculated $I_{sp}$ values must be correlated with experimentally derived data, such as the heat of combustion ($\Delta H_{comb}$) \cite{Azizi2024} or burning rate (r) \cite{Yaman2014}, especially in complex systems containing metallic additives \cite{Wang2025}.
\end{itemize}

\subsection{Catalyst and Formulation Parameters}

\begin{itemize}
    \item Catalyst Identity and Performance
    \begin{itemize}
        \item The type of catalyst significantly determines the thermal response, influencing key metrics like the high-temperature decomposition ($T_{HTD}$) peak \cite{Yadav2021} \cite{Chandrababu2023} \cite{Zhou2025}.
        \item Specific metal additives, such as copper (Cu) and zinc (Zn) nanometals, are highly effective, with Zn achieving the largest reduction in $T_{HTD}$ (up to $93^\circ C$) and Cu yielding the highest heat release \cite{Chandrababu2023}.
        \item Complex catalysts like Fe-based MOFs and $\text{CuCr}_2\text{O}_4$ drastically accelerate decomposition kinetics, reducing the HTD temperature by over $100^\circ C$ and substantially increasing heat release ($\Delta H$) \cite{Guo2025} \cite{Hosseini2019} \cite{Peng2020}.
    \end{itemize}
    \item Physical Parameters
    \begin{enumerate}
        \item Bulk Composition: Propellant performance is highly sensitive to the physical state of the ingredients, particularly particle size and mixture homogeneity \cite{Aziz2015} \cite{Cang2023} \cite{Yadav2021}.
        \item Nanosized Advantage: Nanosized additives (Cu, Ni, Zn) are documented to have superior catalytic activity compared to microsized counterparts due to higher surface area, accelerating exothermic decomposition \cite{Song2008} \cite{Chaturvedi2019} \cite{Yadav2021}.
        \item Intimate Contact: The close physical contact achieved through advanced formulation methods, such as embedding the catalyst (like copper chromite, CC) within the AP particle structure, results in a far more pronounced catalytic effect, altering the decomposition kinetics from the initial Low-Temperature Decomposition (LTD) stage onward \cite{Saha2024} \cite{Hosseini2019}.
    \end{enumerate}
    \item Formulation Variables - Formulation Ratios and Confinement Impact Kinetics: Adjusting the oxidizer-to-fuel ratio (Oxygen Balance, OB) fundamentally changes the thermokinetic parameters and hazard level of the composite \cite{Guo2024}. Furthermore, experimental parameters like increased sample mass (space confinement) can shift the decomposition peak to a lower temperature, influencing measured kinetics \cite{Li2022}.
\end{itemize}

\subsection{Linking Catalysis to Mechanisms via Kinetic Modeling
Synergistic Interactions}

\begin{itemize}
    \item Synergistic Interactions Change Reaction Pathways: The presence of the fuel binder (HTPB) and catalysts (MOFs) demonstrates a synergistic effect, accelerating AP decomposition beyond what is predicted from the components in isolation \cite{Guo2025}. Similarly, the presence of AP (as oxidizer) alters the reaction model of the binder matrix (M3CN/DEGDN), shifting it from sigmoidal kinetics to a nucleation (power law) mechanism ($P1/4$ or $P1/3$) \cite{Boukeciat2025} \cite{Saha2024}.
    
    \item Quantifying Catalytic Effects
    \begin{itemize}
        \item Kinetics Quantify Catalytic Errors: Isoconversional methods (e.g., FWO, Kissinger, Ozawa) provide quantifiable kinetic parameters ($E_\alpha$)  that show the magnitude of the catalyst induced shift \cite{Boukeciat2025} \cite{Guo2024} \cite{Peng2020} \cite{Chandrababu2023}.
        \item Example: In copper-catalyzed systems, the activation energy $(E_\alpha)$ for AP decomposition can be reduced significantly (e.g., from 207 $kJ\cdot mol^{-1}$ to 128 $kJ\cdot mol^{-1}$) due to the specific catalyst \cite{Peng2020} \cite{Chandrababu2023}.
    \end{itemize}
    
    \item Atomic-Scale Mechanisms
    
    \begin{enumerate}
        \item Atomic Simulations Define Fundamental Decomposition: Reactive Molecular Dynamics (MD) is employed to uncover the ultrafast reaction mechanisms, proposing a detailed decomposition network \cite{Chu2023}.
        \item Simulations confirm that the non-catalyzed initial step of AP thermolysis is dominated by proton transfer \cite{Chu2023} \cite{Yadav2021} \cite{Chandrababu2023}
        
        \[
            \text{NH}_4^{+} + \text{ClO}_4^{-} \to \text{NH}_3 + \text{HClO}_4
        \]

        \item Catalysts Drive Mechanisms Toward Electron Transfer: The mechanistic discrepancy between theoretical proton transfer (non-catalyzed) and empirical observation (catalyzed) is explained by the hypothesized electron transfer mechanism, which is favored by the partially filled d-orbitals of transition metal catalysts \cite{Sivadas2019} \cite{Hosseini2019} \cite{Yadav2021}.
        \item Kinetic shift is confirmed by gas analysis, which shows an increased evolution of highly oxidized species $(\text{O}_2, \text{Cl}_2, \text{N}_2\text{O})$ in the presence of copper oxide compared to pure AP \cite{Sivadas2019} \cite{Chandrababu2023}.
        \item The contact achieved by catalyst embedding can lead to the empirical detection of new, highly oxidized reaction intermediates $(\text{HClO}_4, \text{ClO}_3, \text{ClO}_4)$ via Mass Spectrometry reveal species linked to chlorine chemistry pathways that accelerate the overall reaction \cite{Saha2024} \cite{Chu2023}.
    \end{enumerate}
\end{itemize}

\subsection{Research Questions and Proposal Context}

\begin{enumerate}
    \item Catalyst Purpose: Catalytic metal additives (such as copper oxide (CuO), iron oxide $(\text{Fe}_2\text{O}_3)$, or various nanoparticles) are used precisely because they dramatically accelerate the thermal decomposition of AP, enabling high burn rates \cite{Yadav2021} \cite{Vara2019} \cite{Chandrababu2023}.
    \item Kinetic Alterations: These catalysts change the thermal decomposition profile of AP by shifting the high-temperature decomposition (HTD) peak to a much lower temperature and significantly reducing the required activation energy \cite{Yadav2021} \cite{Benhammada2020} \cite{Peng2020} \cite{Li2022}.
    \item Kinetic Alterations: These catalysts change the thermal decomposition profile of AP by shifting the high-temperature decomposition (HTD) peak to a much lower temperature and significantly reducing the required activation energy \cite{Yadav2021} \cite{Benhammada2020} \cite{Peng2020} \cite{Li2022}.
    \item Correlation Hypothesis: This paper hypothesizes that formulations containing a higher relative concentration of catalyst will exhibit larger prediction errors in the equilibrium codes. The increased error is expected to be most visible when the catalyst significantly influences the decomposition temperature.
    \item Discrepancies in Focus: Factors such as catalyst type, particle size, surface area, and crystalline structure all influence catalytic efficiency \cite{Yadav2021} \cite{Rodriguez2019}. These detailed interactions may be the source of predictive difficulty for simple equilibrium models.
    \item Optimal Concentration: Experimental data suggests that an optimal catalyst concentration exists (e.g., between 0.75\% and 1\% $\text{Fe}_2\text{O}_3$) where performance peaks before decreasing due to competing effects \cite{Rodriguez2019}. Such non-linear behavior may translate to varying degrees of predictive errors depending on the loading percentage.
\end{enumerate}

\subsection{Atomic-Scale Simulations and Mechanistic Insights}
\begin{enumerate}
    \item The consensus from atomic-scale research suggests that these simulations are vital for uncovering the kinetic pathways missing from macroscopic models, making them potentially detailed enough to explain macroscopic prediction errors.
    \item Need for Mechanisms: The detailed reaction network and exact mechanisms involved during AP thermal decomposition are considered the "missing piece" in high-fidelity combustion models \cite{Chu2023}. Simple models struggle particularly when energetic additives or catalysts are introduced.
    \item LAMMPS Role (Hypothesis): The core hypothesis of this proposal is that LAMMPS molecular dynamics simulations using the ReaxFF force field will successfully reveal alternative reaction pathways in catalyzed systems . These alternative routes represent kinetically-controlled mechanisms that deviate from the standard equilibrium path, thus explaining why equilibrium-based thermochemical codes fail to accurately predict the final macroscopic performance metrics like $I_{sp}$.
    \item Bridging the Scales: By identifying these atomic-level pathways, the research aims to bridge the disconnect between molecular-atomic mechanisms and macroscopic performance predictions, establishing a relationship currently absent in the literature.
\end{enumerate}

\subsection{Overview of Propellant Literature Sources}

\subsubsection{Core Propellant Composition and Fundamentals}

\begin{enumerate}
    \item Ammonium Perchlorate (AP) based solid propellant is a modern standard used in various applications and is the predominant oxidizer in solid rocket propulsion systems. They offer advantages such as good specific impulses and high reliability. The AP/HTPB combination is the primary choice for most contemporary missiles and rockets.
    \item Solid propellants typically consist of a polymeric binder, an oxidizer (AP), metal fuel (e.g., aluminum (Al)), and curing agents. Propellants generally contain at least ten different ingredients.
    \item Binder Role and Types: The binder (HTPB, CTPB, PBAN) provides the structural matrix or "glue" holding solid components and functions as a fuel. HTPB is recognized for offering good mechanical properties and aging characteristics. Curing: HTPB can be cured using isocyanates, such as Toluene diisocyanate (TDI) or Isophorone diisocyanate (IPDI).
    \item Manufacturing Complexity: The manufacturing process that produces the solid propellant significantly influences the quality of the finished grain. Manufacturing may involve at least 30 steps.
\end{enumerate}

\subsubsection{Catalytic Additives and AP Decomposition Mechanisms}

\begin{enumerate}
    \item Catalyst Necessity: Transition Metal Oxides (TMOs) are the primary means to achieve high burn rates, as the thermal decomposition of AP is highly sensitive to the presence of certain additives. Catalysts are typically incorporated at low percentages (e.g., 0.1 wt\% - 2 wt\%).
    \item Mechanism of Action: Catalysts reduce the required activation energy $(E_\alpha)$ and enhance the reaction rate. Fe-based coordination polymers and active metal centers serve as intermediates for electron transfer between $\text{ClO}_4^{-}$ and $\text{NH}_4^{+}$ during the low temperature decomposition (LTD) process. 
    \item Nano-Effects: Nano-sized catalysts (e.g., Fe, Co, Ni, Cu, Zn, or their oxides/alloys) exhibit significantly better catalytic activity compared to bulk materials due to their large surface area and small particle size.
    \item Specific Decomposition Reductions: The optimal composition of 77\% AP/23\% sorbitol, with metal additives (Magnesium or ferrous oxide), lowers the AP thermal decomposition temperature to 210.38°C.
    \item Ferric oxide addition (0.05\%) lowered the AP/sorbitol ignition temperature to 199.15°C and merged two exothermic peaks into a single peak (Azizi et al., 2025).
    \item $\text{Cu}_3(\text{BTC})_2$ reduced the AP high-temperature decomposition (HTD) peak temperature from $442^\circ C$ to around $340^\circ C$ and lowered the apparent $E_\alpha$ from 207 $kJ\cdot mol^{-1}$ to approximately 128 $kJ\cdot mol^{-1}$ \cite{Peng2020}.
    \item Synergistic Effects: A synergistic effect between Fe-based MOFs and HTPB substantially accelerates the decomposition of both components. In AP/HTPB/MOFs composites, this synergy reduced the HTD temperature from $437.2^\circ C$ to $342.7^\circ C$ \cite{Guo2025}.
    \item Proximity and Concentration: The catalytic effect on the LTD of AP is enhanced when the catalyst (e.g., Copper Chromite (CC)) is in close contact or "embedded" within the AP \cite{Saha2024} \cite{Maro}(Saha et al., 2024; Marothiya et al., 2017). Experimental findings suggest an optimal catalyst concentration exists (e.g., between 0.75\% and 1\% $\text{Fe}_2\text{O}_3$). WHERE DID I GET THIS SOURCE FROM?!?! for peak performance, as higher loadings may decrease performance due to the catalyst soaking heat away from the reaction \cite{Rodriguez2019}.
\end{enumerate}

\subsubsection{Aluminized Propellants and Condensed Combustion Products (CCPs)}

\begin{enumerate}
    \item Aluminum (Al) is commonly added to propellants to enhance performance by increasing combustion temperature. However, the formation of large aluminum agglomerates in Condensed Combustion Products (CCPs) is detrimental, reducing the specific impulse ($I_{sp}$).
    \item Particle Size Influence: Increasing the particle size of AP (e.g., to 400-500 $\mu$m) is generally helpful in reducing the size of CCPs and decreases Al particle agglomeration \cite{Tu2024}.
    \item Modeling Accuracy: Predicting agglomerate size requires models beyond simple empirical or classical pocket methods. The Secondary Agglomeration Model ($D_{agg,2}$) provides better agreement with experimental measurements (within 10\% relative error) for aluminized NEPE propellants, highlighting the significance of secondary agglomeration \cite{Tu2024}.
    \item Coated Metal Fuels: Coating micron Al ($\mu$Al) with bimetallic nanoparticles ($nCu+nNi/\mu Al$) accelerates the ($\mu$Al) oxidation reaction by nearly $400^\circ C$ and significantly enhances AP thermal decomposition \cite{Wang2025} \cite{Song2008}.
\end{enumerate}

\subsubsection{Computational Modeling and Kinetics}

\begin{enumerate}
    \item Thermochemical Tools: Computational tools like PROPEP 3.0 and NASA-CEA2 are widely utilized to assess the theoretical performance $(I_{sp})$ and feasibility of propellant formulations based on the principle of thermochemical equilibrium.
    \item Mesoscale Simulation: Partitioned numerical frameworks have been developed to simulate heterogeneous combustion at the mesoscale (e.g., AP/HTPB packs), successfully predicting burning rates over a wide pressure range in agreement with experimental data \cite{Cang2023} \cite{Wang2025}.
\end{enumerate}

\subsubsection{Concluding Gap}
\begin{itemize}
    \item The scholarly context reveals an understanding of individual catalytic mechanisms and advanced modeling techniques, but lacks a unifying validation layer, which this proposal aims to provide \cite{Li2022}.
    \item While computation tools exist and have been developed for decades (since the late 20th century), their general reliability is poorly documented across diverse AP-based formulations containing catalytic metal additives. Current literature provides only isolated comparison data sets rather than systematic, correlated validation across multiple software packages (NASA-CEA2, PROPEP 3.0, CPROPEP, RPA, GUIPEP).
    \item Literature extensively covers how catalysts modify decomposition kinetics (e.g., reducing $E_\alpha$) \cite{Yadav2021}, but these studies rarely quantify how these microscopic changes affect the prediction accuracy of macroscopic performance metrics ($I_{sp}$, $\Delta H^\circ_{comb}$) calculated by equilibrium codes.
\end{itemize}

\subsubsection{The Bridge via LAMMPS/ReaxFF}

\begin{itemize}
    \item This proposal aims to bridge this gap by employing atomic-scale molecular dynamics simulations (LAMMPS w/ ReaxFF) to model reaction pathways, generating insights that can explain the variance (or "missing piece") between equilibrium-based predictions and empirical observations \cite{Li2022} \cite{Chu2023}.
    \item The ultimate goal is to generate a correlative framework that provides quantitative guidance documentation and confidence thresholds for computational tool trustworthiness, enabling propellant researchers to rely more confidently on digital screening.
\end{itemize}

\newpage
\printbibliography

\end{document}

\end{document}

