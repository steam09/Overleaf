\documentclass[11pt,letterpaper]{article}


\usepackage{graphicx}
\usepackage{xcolor}

% For rounded framed boxes to highlight citations
\usepackage[framemethod=tikz]{mdframed}

% Lorem Ipsum Parargraph
\usepackage{lipsum}

\usepackage[english]{babel}
\usepackage{csquotes}
\usepackage[
  backend=biber,
  style=ieee,
  sorting=none,
  maxcitenames=1,     % In-text: 1 author + et al.
  mincitenames=1,
  maxbibnames=1,      % Bibliography: 1 author + et al.
  minbibnames=1
]{biblatex}
\addbibresource{references.bib}


% Additional packages for formatting
\usepackage[margin=1in]{geometry}
\usepackage{enumitem}
\usepackage{titlesec}
\usepackage{hyperref}
\usepackage{pifont}

% Checkbox setup
\newlist{todolist}{itemize}{2}
\setlist[todolist]{label=$\square$}
\newcommand{\cmark}{\ding{51}}%
\newcommand{\xmark}{\ding{55}}%
\newcommand{\done}{\rlap{$\square$}{\raisebox{2pt}{\large\hspace{1pt}\cmark}}%
\hspace{-2.5pt}}
\newcommand{\wontfix}{\rlap{$\square$}{\large\hspace{1pt}\xmark}}

\titleformat{\section}{\large\bfseries}{\thesection.}{0.5em}{}
\setlength{\parindent}{0pt}
\setlength{\parskip}{0.5em}

\begin{document}

\begin{center}
    {\LARGE\bfseries STEM/AP Research}\\[0.3em]
    {\Large Inquiry Proposal Form}
\end{center}

\vspace{1em}

\textit{In conjunction with the other forms provided, complete each of the items below prior to inquiry proposal submission. Ensure that all materials submitted are accurate, consistent, and reflective of the professional quality of your work.}

\vspace{1em}

\section{DONE State your research question and/or project goal(s).} % Question 1

\textbf{Primary Question:} Can regression-based correction factors derived from LAMMPS simulations improve the predictive accuracy of PROPEP/NASA-CEA2 outputs for catalyzed AP systems?

\textbf{Main Goals:}
\begin{itemize}
    \item Quantify / model the deviations between equilibrium-based predictions and empirical results for catalyzed AP formulations
    \item Derive kinetic and mechanistic parameters from atomistic (ReaxFF) simulations to differ non-equilibrium correction factors applicable to thermochemical codes
    \item Develop a predictive correction model or reliability index (Computational Reliability Index, CRI) that relates formulation characteristics (catalyst identity, concentration, particle size) to computational accuracy
    \item Validate the proposed/developed framework through a conjunction between meta-analysis and cross-software benchmarking under standardized conditions
\end{itemize}

In other words, this research is planning to test current simulation softwares against empirical data done in a professional setting, and correlating the effect of catalysts on error patterns. Then, molecular dynamics simualations will be used to derive kinetic parameters to explain any discrepencies, mainly through the non-equilibrium mechiansms that catalysts can introduce. Finally, a hybrid framework will be created in a correlation constant / index form to identify which basic compound types can be trusted to depend on certain softwares. This data will not cover all formulations, but rather the most common ones / already research compounds in literature.

Independent variables (IV) such as formulation characteristics (oxidizer/fuel ratio, particle sizes, catalyst concentration) will all be used and tested against the dependent variables (DV) of computational accuracy to identify patterns. This study is a multivariate regression analysis at its core. 

\section{DONE Identify the reasons for choosing the topic of interest and research question/project goal(s).} % Question 2

\begin{itemize}
    \item In the practice of chemical and materials engineering, computational tools have been in development for these exact concepts for years, but those same softwares, as elaborated later, rely on equilibrium thermochemistry principles that may not hold true for catalyzed systems.
    \item A significant disconnect exists between atomistic simulations, empirical data, and the codes behind the softwares that engineers use. While there are many papers that have used these same tools, with a spike in recent years, the equilibrium  assumptions that are made fail to properly capture the non-equilibrium kinetics and mechanisms introduced by catalysts.
    \item The ability to depend on these softwares with enough confidence can accelerate the development of new and improved formulations, assuming enough documentation and validation exists. These goals are especially aimed at groups who are unable to provide the testing that larger companies can replicate, either due to cost, safety, or time-based constraints. 
    \item Personal Reasons: Experience in materials science (namely fuel propulsion), programming (Python, R, MATLAB), AP Chemistry, and computational simulation tools. 
\end{itemize}

\newpage

\section{DONE ENOUGH Provide a brief annotated list of three key studies that have informed your understanding of the scholarly conversations surrounding your topic. Describe what each has contributed to the development of your inquiry proposal.} % Question 3

\begin{enumerate}
    \item \textbf{Azizi et al. (2025). Heat of Combustion and Thermal Decomposition of Ammonium Perchlorate/Sorbitol Solid Propellant with Metal Additives}
    
    This study talks about how to calculate theoretical $I_{sp}$ of different AP/Sorbitol formulations containing the additives using PROPEP 3.0. Comparing those theoretical $I_{sp}$ values with the empirical ones results in the heat of combustion through calorimetry. It addresses the challenge that computational equilibrium codes might fail to capture the full energetic benefit of additives, which necessitates the need for other softwares and an atomic-scale simulation. Findings suggest that the addition of metal catalysis increases the heat of combustion, while the $I_{sp}$ is still unclear in correlation, which could be interesting to talk about in patterns and why they may have occurred.
    
    \item \textbf{Yadav et al. (2021). Recent advances in catalytic combustion of AP-based composite solid propellants}
    
    This review establishes the basic observations (qualitative) regarding how additives influence performance, another source for empirical data. Yadav focuses mainly on the transition metal oxides, through how they reduce activation energy (see in LAMMPS), High-Temperature Decomposition, and heat release over time. It shapes the second question in how some pattern errors might correlate to specific catalysis types/concentrations, which might be dependent on particle size, surface area, and/or crystalline structure formations.
    
    \item \textbf{Chu et al. (2023). A reaction network of AP decomposition: the missing piece from atomic simulations}
    
    This research provides good methodology for the atomic-scale component of the secondary questions, which is using a Neural Network Potential for decomposition kinetics. It provides the reaction network, mechanisms in simpler models, and other transfers that are not always identified when catalysts are introduced. These findings can reveal some plausibility on why combinations could induce variance, and develop the proposal to introduce this ``missing piece'' for advanced-systems combustion modeling.
\end{enumerate}

\newpage

\section{DONE Identify the gap addressed by your proposed research. Explain how the gap is situated into the scholarly situation. Provide sources to justify the gap your proposed research is addressing.} % Question 4

While computational tools exist for propellant development since the late 20th century, their predictive ability is still not completely explored and poorly documented across many types of combustion formulations (in this case AP-based ones containing metal additives). No cross-tool validation framework exists that is able to evaluate which computational approaches are most effective under different conditions.

\textbf{Scholarly Context}

There are three main research streams:
\begin{enumerate}
    \item \textbf{Computational Tool Development} --- Developed tools (such as RPA, CPROPEP, etc.), but they all use similar equilibrium thermochemistry principles. However, they have isolated comparison data sets rather than correlated validation across multiple sets.
    
    \item \textbf{Catalytic Additive Research} --- Much literature document how transition metal oxides and nanoscale metals modify AP decomposition by altering pathways and activation energies. But, the studies focus on a mechanistic understanding rather than how these changes affect computational accuracy.
    
    \item \textbf{Atomic-Scale Modeling} --- Advances in NNP/MD simulations provide details about decomposition, but integration between traditional performance prediction tools is still mediocre. Connections between molecular-atomic level mechanisms and macroscopic metrics are not properly established.

\end{enumerate}

Current research lacks a method to correct these errors based on first-principles chemistry. This paper aims to introduce a hybrid framework that uses ReaxFF molecular dynamics to derive atomistically-informed correction factors, resulting in a Computational Reliability Index (CRI) to guide model trustworthiness.

\newpage

\section{DONE Describe your chosen or developed research method and defend its alignment with your research question.} % Question 5

KEY TERMS:
\begin{itemize}
    \item ReaxFF: Reactive Force Field; 
    \item LAMMPS: Large-scale Atomic/Molecular Massively Parallel Simulator;
    \item NNP: Neural Network Potential
    \item Machine Learning Model: Random Forest Regression to learn patterns from data in making future Predictions
    \item Computational Reliability Index (CRI): A confidence score (0, 1) in a researcher's ability to "trust" simulation predictions (think of credit scores, but less granular). 
\end{itemize}


This study will include a hybrid methodology combining meta-analysis, equilibrium thermochemistry, atomistic simulation, and statistical modeling. The research will progress through four main integrated phases:

\textbf{Phase 1 - Literature Meta-Analysis}
\begin{itemize}
    \item A systematic extraction of empirical performance data ($I_{sp}$, $\Delta H_{comb}^{\circ}$, decomposition kinetics) from peer-reviewed journals, technical reports (NASA, AIAA, AFRL), and materials databases
    \item Compilation of 25-30 distinct AP-based formulations spanning diverse catalyst systems (Cu,\newline Fe2O3, Mg, CuO, NiO) at varied concentrations
    \item Documentation of experimental conditions (chamber pressure, temperature, particle size) to enable standardized computational comparison
    \item Creation of validated benchmark dataset against which computational predictions can be evaluated
\end{itemize}

\textbf{Phase 2 - Software Benchmarking}
\begin{itemize}
    \item Generation of theoretical performance metrics for each benchmark formulation using four equilibrium-based tools: rocketCEA, RPA Lite, Cantera, PROPEP 3.0
    \item Standardization of input conditions across all tools - 1000 psi chamber pressure, 298 K initial temperature, sea-level expansion ratio
    \item Recording of predicted $I_{sp}$, combustion temperature $T_c$, $\Delta H_{comb}^{\circ}$, and exhaust composition
    \item Calculation of accuracy metrics ($R^2$, RMSE, MAPE) comparing predictions to empirical values
    \item Analysis by catalyst type, concentration, and formulation complexity to single-out error patterns
\end{itemize}

\textbf{Phase 3 - Molecular Dynamics Simulations}
\begin{itemize}
    \item Selection of representative formulations (5-7) from the benchmark set for detailed atomic-scale analysis
    \item LAMMPS Simulation using ReaxFF force fields to model thermal decomposition of AP with/without catalysts
    \begin{itemize}
        \item NPT ensemble equilibration at 298 K for 50–100 ps
        \item Production will run for 2-5 nanoseconds to capture decomposition events
        \item ReaxFF is chosen for its ability to:
        \begin{enumerate}
            \item Simulate dynamic bond/formation/breaking in real reactions
            \item Configure 5k - 10k atoms over the 5 nanoseconds ran (sufficient enough to ignore ab. initio method that have quantum accuracy)
            \item Previous force fields validated for AP and metal oxides in established literature
        \end{enumerate}
        \item Neural Network Potentials (NNP), while more accurate, lack the validation to be used for extensive research, and is beyond the scope of this project
    \end{itemize}
    \item Extraction of kinetic parameters from trajectories: activation energies ($E_a$), pre-exponential factors (A), decomposition rate constants
    \item Identification of reaction pathways, intermediate species populations, and mechanistic deviations from uncatalyzed systems
    \item Quantification of non-equilibrium effects induced by catalytic additives
\end{itemize}

\textbf{Phase 4 - Multivariate Machine Learning}
\begin{itemize}
    \item Integration of LAMMPS-derived kinetic parameters with formulation descriptors (catalyst identity, concentration, particle size, oxidizer-to-fuel ratio) as input features
    \item Training of multivariate regression model (Random Forest) to predict magnitude of computational error for equilibrium-based tools. A potential future work could be using Gaussian Process Regression, but that is complexity without proportional benefit to justify the time committment. 
    \item Feature importance analysis (e.g., SHAP values) to identify formulation variables most \newline strongly influence predictive ability
    \item Potentially developing a Computational Reliability Index (CRI) with:
    \begin{itemize}
        \item CRI $\le$ 0.8: Computational predictions reliable for design decisions
        \item 0.5 $\le$ CRI $\le$ 0.8: Predictions suitable for preliminary screening, caution advised
        \item RI $\le$ 0.5: Empirical validation required due to kinetically-controlled effects
    \end{itemize}
\end{itemize}

\newpage

\section{DONE What additional approval processes are required for your research? Select all that apply.} % Question 6

\ding{113} Human subjects [requires additional IRB review and approval if student wants to publish and/or publicly present]

\ding{113} Animal subjects [requires additional review or approval by school or district processes]

\ding{113} Harmful microorganisms [requires additional review or approval by school or district processes]

\ding{113} Hazardous materials [requires additional review or approval by school or district processes]

\ding{110} No additional review or approvals required

\section{DONE Explain how your proposed method complies with ethical research practice.} % Question 7

No human subjects will be at risk of data-leaks, but rather only free-access sets and researcher-side computational testing will be applied.

\newpage

\section{DONE Describe the data or additional scholarly work that will be generated to answer your proposed research question or achieve your project goal.} % Question 8

\begin{enumerate}
    \item \textbf{Benchmark Dataset}
    \begin{itemize}
        \item Database (personally compiled) of AP-based propellant formulations from published, open-access literature
        \item Empirical performance metrics ($I_{sp}$, seconds) and heat of combustion ($\Delta H_{comb}^{\circ}$, kJ/kg)
        \item Catalyst types, concentrations, oxidizer/fuel ratios, and particle sizes
        \item Experimental conditions: chamber pressure, temperature, expansion ratio, calorimetry methods
        \item Target: 25-30 distinct formulations over catalyst systems
    \end{itemize}
    
    \item \textbf{Computational Predictions}
    \begin{itemize}
        \item Predicted $I_{sp}$ and $\Delta H_{comb}^{\circ}$ for each formulation using four tools: NASA-CEA2, PROPEP 3.0, CPROPEP, and RPA Lite
        \item Secondary outputs: combustion temperature ($T_{ad}$), exhaust composition, characteristic velocity ($c^*$)
        \item Standardized input conditions applied consistently across all tools
        \item Deviation calculations: percent error, absolute error for each combo
    \end{itemize}
    
    \item \textbf{Accuracy Metrics}
    \begin{itemize}
        \item Overall tool performance: correlation coefficients ($R^2$), root mean square error (RMSE), mean absolute percentage error (MAPE)
        \item Accuracy analysis by catalyst type (Fe$_2$O$_3$, Cu, Mg, etc.), concentration ranges (0–10\% by weight), \& formulation complexity
        \item Comparative analysis identifying which tools perform best under specific conditions
        \item Statistical significance testing (ANOVA) for differences between tool accuracies
    \end{itemize}
    
    \item \textbf{Molecular Dynamics Simulations}
    \begin{itemize}
        \item LAMMPS trajectory files for 5–8 representative formulations (uncatalyzed baseline, high-performers, high-error cases)
        \item Extracted kinetic parameters: activation energies ($E_a$, kJ/mol), pre-exponential factors ($A$), decomposition rate constants
        \item Reaction pathway diagrams identifying bond-breaking sequences and intermediate \newline species formation
        \item Quantification of mechanistic deviations: comparison of catalyzed vs. uncatalyzed decomposition mechanisms
        \item Time-resolved species populations and energy release profiles
    \end{itemize}
    \item \textbf{Hybrid Correction Framework \& CRI}
    \begin{itemize}
        \item Machine learning model (Random Forest or Gaussian Process Regression) predicting computational tool error as a function of formulation characteristics and atomistic kinetic parameters
        \item Model validation metrics: cross-validation performance, MAPE reduction relative to uncorrected predictions, generalization testing on withheld formulations
    \end{itemize}
    
    \item \textbf{Decision Framework Documentation}
    \begin{itemize}
        \item Workflow diagrams illustrating when to apply specific computational tools based on formulation characteristics
        \item Best practices guide for propellant researchers: tool selection recommendations, reliability limitations, and validation requirements
        \item Case studies demonstrating CRI application to novel formulations not in the training dataset
    \end{itemize}
\end{enumerate}

\textbf{Visualizations}
\begin{itemize}
    \item Scatter plots: predicted vs. empirical values with regression lines and confidence intervals for each tool
    \item Heat maps: accuracy metrics ($R^2$, MAPE) across catalyst type and concentration matrices
    \item Error distribution histograms: tool-specific and formulation-specific prediction error patterns
    \item Molecular visualizations: reaction pathway diagrams from LAMMPS trajectories showing catalyzed vs. uncatalyzed mechanisms
    \item Feature importance plots: SHAP value summaries identifying key drivers of prediction accuracy
    \item CRI distribution plots: confidence score distributions across the benchmark dataset
    \item Tool comparison radar charts: multi-dimensional performance comparisons across accuracy metrics
\end{itemize}

\newpage

\section{Describe the way you will analyze the data or additional scholarly work generated by your method and justify its alignment with your research question and/or project goal.} % Question 9

\section{Describe the way you will analyze the data or additional scholarly work generated by your method and justify its alignment with your research question and/or project goal.} % Question 9

\subsection*{Statistical Design Clarification}

This project uses a multivariate observational research design to evaluate predictive abilities of softwares with prediction error magnitudes (MAPE) as the dependent variable. Multiple independent variables, as stated earlier, will be uesd as predictors within the Random Forest regression model.

\subsection*{Overview of Analytical Tools Development}

\textbf{Tool 1: Automated Prediction Pipeline}
\begin{itemize}
    \item \textit{Purpose:} To generate predictions from the four thermochemical tools for each formulation in literature
    \item \textit{Method:} Scripts (python) to standardize input generation, executing simulation through softwares (GUI or command-line), and providing outputs in a common format (most likely CSV)
    \item \textit{Validation:} To compare the outputs against known results 
\end{itemize}

\textbf{Tool 2: LAMMPS Analysis Workflow}
\begin{itemize}
    \item \textit{Purpose:} Extract kinetic parameters (Ea, pre-exponential factors) from molecular dynamics trajectories
    \item \textit{Method:} Developed scripts to track compound relative ratios over time and fit to Arrheniun equations
    \item \textit{Validation:} Pure AP simulation can reproduce literature activation energies (150-180 kJ/mol)
\end{itemize}

\textbf{Tool 3: Random Forest Regression Model}
\begin{itemize}
    \item \textit{Purpose:} Predict computational tool error based on formulation characteristics through controlled learning
    \item \textit{Method:} Train scikit-learn Random Forest model with 80/20 train-test split and 5-fold cross-validation
    \item \textit{Validation:} Test predictions on withheld formulations to evaluate general performance (MAPE)
\end{itemize}

\textbf{Tool 4: Computational Reliability Index (CRI)}
\begin{itemize}
    \item \textit{Purpose:} Provide developers/researchers with a confidence score for computational predictions
    \item \textit{Method:} Transform Random Forest predicted error into 0-1 scale with interpretable thresholds (>0.8 = reliable, <0.5 = validation needed)
\end{itemize}

\subsection*{Statistical Analysis Approach}

\textbf{Descriptive Statistics:}
\begin{itemize}
    \item Calculate accuracy metrics (R², RMSE, MAPE) for each tool
    \item Create visualizations: scatter plots (predicted vs. empirical), error distributions, heat maps (error by catalyst type / tool)
\end{itemize}

\textbf{Comparative Analysis:}
\begin{itemize}
    \item ANOVA in testing tool accuracy == differs significantly
    \item Analysis comparing uncatalyzed vs. catalyzed system performance
\end{itemize}

\textbf{Multivariate Modeling:}
\begin{itemize}
    \item Random Forest regression w/ prediction error value as target variable (DV)
    \item Input features: catalyst type, concentration, particle size, tool selection, LAMMPS-derived kinetic parameters
    \item Feature analysis to identify variables that most strongly influence accuracy
\end{itemize}

\textbf{Mechanistic Correlation:}
\begin{itemize}
    \item Compare LAMMPS-derived activation energies (E_a) to empirical literature
    \item Test if larger deviations from equilibrium assumptions correlate with higher prediction errors for software deviations
\end{itemize}

\subsection*{Analytical Software}
Python (pandas, scipy, scikit-learn, matplotlib) for data manipulation, statistical testing, machine learning, and visualization. LAMMPS native tools and VMD for trajectory analysis. Repeated in above section

\subsection*{Alignment with Research Question}
This analytical framework addresses the intended research question by:
\begin{enumerate}
    \item Quantifying current tool accuracy through testing
    \item Developing a model that incorporates atomistic kinetic data as inputs
    \item Producing a reliability index that can be uses before committing to experimental testing (time and cost savings)
    \item Demonstrating whether atomic-scale mechanisms explain macroscopic prediction failures
\end{enumerate}

\newpage

\section{DONE List any equipment, resources, and permissions needed to collect data or information. Attach the initial drafts that apply to your proposal if engaged in human subject research: informed consent forms; surveys, interview questions, questionnaires, or other data collection devices; or letters/flyers that will be distributed to study subjects.} % Question 10

\textbf{Hardware}
\begin{itemize}
    \item Personal laptop for programs, data analysis, and literature review
    \item School-based computer for extended simluations should time permit
    \item Cloud computing platforms (mainly Google Colab) if needed for machine learning model training
\end{itemize}

\textbf{Software}
\begin{itemize}
    \item rocketCEA - Python wrapper of NASA-CEA2 code
    \item RPA Lite - Rocket propulsion analysis software
    \item Cantera - Open-source suite for chemical kinetics, thermodynamics, and intermediates
    \item PROPEP 3.0 - Propellant evaluation program from AFRL
    \item If needed, use Google Colab, AWS, or Jupyter Notebooks for computing power
    \item SciKit-Learn w/ pandas: one for data cleaning and organizing, one for training the model
\end{itemize}

\textbf{Molecular Dynamics Sims}
\begin{itemize}
    \item LAMMPS - open-source MD simulator
    \item ReaxFF force fields validated for AP and metal oxides
    \item Visualization tools: VMD (Visual Molecular Dynamics) and/or OVITO in trajectory analysis
\end{itemize}

\textbf{Data Analysis}
\begin{itemize}
    \item Python w/ libraries: pandas, scripy, scikit-learn, numpy, matplotlib/seaborn, shap
\end{itemize}

\textbf{Development Environments}
\begin{itemize}
    \item Jupyter Notebooks - Ease of access reproducible workflows
    \item Google Colab
    \item Git/Github - Version control, code backup, documentation, and storage for open source access
\end{itemize}

\textbf{Literature Access}
\begin{itemize}
    \item Google Scholar, ResearchGate, NASA Technical Reports, JSTOR, ScienceDirect, other journals
\end{itemize}

No permissions from IRB, laboratory approval, or database access needed. No attached forms for human subjects.

\newpage

\section{DONE Describe the anticipated logistical and personnel challenges for your research project (to collect and analyze data or pursue research methods appropriate to a paper that supports a performance/exhibit/product).} % Question 11

\begin{enumerate}
    \item \textbf{Literature Data Availability}
    \begin{itemize}
        \item Inability to find 30+ formulations for the criteria, or uneven catalyst/concentration coverage
        \item Lower \# of formulations (20--25), expand additive types, or pre-2000s sources (with skepticism)
    \end{itemize}
    
    \item \textbf{LAMMPS Technical Curve}
    \begin{itemize}
        \item Steep learning curve, convergence issues, computational resources (laptop)
        \item Potential changes: smaller simulation atoms, \# of representative formulations, cloud computing, shorter simulation times, or a focus on qualitative mechanisms over quantitative kinetics
    \end{itemize}
    
    \item \textbf{Literature Reporting}
    \begin{itemize}
        \item Conditions, units, or other uncertainties
        \item Data quality tiers; sensitivity to lower-``quality'' data; non-parametric tests
    \end{itemize}
    
    \item \textbf{Time Management}
    \begin{itemize}
        \item Ambitious scope vs. limited time, especially for the 4 months provided in a high-school level
    \end{itemize}
\end{enumerate}

\section{Provide a brief timeline that outlines your process from now through project completion.} % Question 12

\begin{enumerate}
    \item 
\end{enumerate}

\section{Discuss the anticipated value and/or broader implications of your research project.} % Question 13

\textbf{Scientific Value:} Validation methodology for computational propellant design tools; linking atomic-scale mechanisms to macroscopic prediction accuracy

\textbf{Practical Impact:} Accelerated development of formulations, fewer iterations to save materials, and less hazardous materials overall

\textbf{Broader Implications:} Simpler for smaller organizations to develop cheaper formulations, especially with limited materials

\textbf{Limitation:} Cannot replace physical empirical validation, but enables pre-screening for non-satisfactory candidates

\vspace{2em}

\textbf{Review team feedback:}

\vspace{2em}

\textbf{STEM/AP Research Teacher's Approval (signature):} \rule{6cm}{0.4pt}

\vspace{1em}

\textbf{IRB Approval Date:} \rule{6cm}{0.4pt}

\vspace{1em}

\textbf{Mentor/Expert Adviser(s):} \rule{6cm}{0.4pt}

\end{document}