
\section{Method \& Design}

\subsection{General Approach}
This research will employ a \textbf{comparative meta-analytical methodology}, which combines computational prediction with a systematic literature review of current datasets. Rather than conducting real-world synthesis and testing (which would require resources, safety, and time outside of the allotted given), this study aims to leverage the pre-existing extensive body of published empirical data from laboratories worldwide to develop performance metrics. These benchmarks are then going to be compared against the predictions generated by a multitude of computational tools, which enable an assessment of prediction accuracy across the formulation types. 

This method of analysis is far optimal for several reasons. First, it allows for a dependence upon professional equipment under controlled supervision, providing a more reliable benchmark than a high-school setting. On top of this, it uses current literature spanning diverse catalysts (30+ formulation)—a sample size that is near impossible to achieve in a high school laboratory within reasonable budget constraints. Second, the encompassing nature of a larger simulation allows for a discrepancy that other studies hold: the need for individualized limitations which holds back the whole. Finally, this methodology is reproducible at any level, compared to the difficulty of consistency between an amateur such as myself relative to a graduate student. 

On top of this, the integration of atomic-scale dynamics simulations (LAMMPS + ReaxFF-AP) with traditional thermochemical equations can represent a multi-scale modeling approach that basic simulations cannot. The combination of bulk property prediction and kinetic molecular digitization provides complementary perspectives in a coordinative way. 

\subsection{Data Sources: Criteria}
\subsubsection{Primary Sources}
Just as a hands-on approach would produce empirical data, in a meta-analysis study the benchmark will be set from external peer-reviewed scientific literature, reports, and databases. Such sources would include:
\begin{itemize}
    \item Academic Journals: Journal of Propulsion and Power, Combustion and Flame, Propellants Explosives Pyrotechnics, Thermochimica Acta, Journal of Thermal Analysis and Calorimetry 
    \item Technical Reports: NASA Technical, AIAA Conferences, AFRL, DRDL
    \item Databases: ResearchGate, ScienceDirect, Google Scholar, JSTOR, SpringerMaterials
\end{itemize}

\subsubsection{Inclusion / Exclusion}
Literature will have the specific criteria to match should they be included in this study:
\begin{itemize}
    \item Performance Metrics: Must report the following metrics: $I_{sp}$, $\Delta H^{\circ}_{\text{comb}}$, or other points to derive these two values. 
    \item Formulation Detail: Detail catalyst identity \& concentration with a general precision for computational replication
    \item Experimental Conditions: Just as in real-world scenarios, we cannot predict all situations in which these propellants might be used. But, within the literature, external conditions need to be stated for consistency, and this includes but is not limited to: chamber pressure, initial temperature, and the calorimetry method. 
\end{itemize}

\subsection{Design: Computation Matrix}
\subsubsection{Thermochemical Systems}
Within this study, there will be the main five computational tools that will general generate predictions with information provided by the dataset:
\begin{enumerate}
    \item NASA Chemical Equilibrium with Applications (CEA2)
        \begin{itemize}
            \item Standard: Gibbs free-energy minimization for equilibrium formulations
            \item Inputs: Chemical formula, chamber pressure, init. temperature
            \item Outputs: $I_{sp}$, combustion temperature, exhaust composition, $\Delta H^{\circ}_{\text{comb}}$
        \end{itemize}
    \item PROPEP 3.0 (ProPepMain 3)
        \begin{itemize}
            \item Standard: Equilibrium dependency with combustion products
            \item Inputs: Compositions \& chamber conditions
            \item Outputs: Theoretical $I_{sp}$, chamber temperature, performance at expansion ratios
        \end{itemize}
    \item CPROPEP (GNU-Octave Script)
        \begin{itemize}
            \item Standard: Equilibrium calculations using NASA-CEA algorithms
            \item Inputs: Composition, chamber pressure
            \item Outputs: $I_{sp}$, $\Delta H^{\circ}_{\text{comb}}$
, equilibrium composition
        \end{itemize}
    \item RPA Lite (Rocket Propulsion Analysis)
        \begin{itemize}
            \item Standard: Thermochemical analysis using nozzle-design tools
            \item Inputs: Formulation, motor geometry
            \item Outputs: $I_{sp}$, characteristic velocity ($c^*$), thrust coefficient
        \end{itemize}
    \item GUIPEP
        \begin{itemize}
            \item Standard: Thermochemical equilibrium
            \item Inputs: Formulation, percentages, operating conditions
            \item Outputs: $I_{sp}$ and $\Delta H^{\circ}_{\text{comb}}$
        \end{itemize}
\end{enumerate}

If not mentioned directly, typical standard conditions are: chamber pressure @ $1000\text{ psi}$, initial temperature @ $298\text{ K}$, expansion @ $1\text{ atm}$ or sea-level. 

\subsubsection{Atomic-Scale Molecular Dynamics Simulations}
\textbf{LAMMPS} (Large-Scale Atomic/Molecular Massively Parallel Simulator) will be the main system for finding mechanistic studies of the formulations. 

\textbf{Simulation Protocol:} To keep consistency between parameters and direction of steps within a reaction. 
\begin{enumerate}
    \item \textbf{Force Field}: ReaxFF reactive potential for AP systems (specific parameter in development as literature review progresses of AP MD studies). 
    \item \textbf{Equilibration}: NPT ensemble (constant particle number, pressure, temperature) at $298\text{ K}$ for $50-100\text{ ps}$.
    \item \textbf{Heating Protocol}: Gradual temperature change from $298\text{ K}$ to $600\text{ K}$ to simulate real thermal decomposition. 
    \textbf{Analysis}: Identifying reaction pathways \& potential differences, calculating energy barriers, and tracking intermediate step populations for input/output. 
\end{enumerate}

\subsection{Variables}
\textbf{Independent Variables:}
\begin{itemize}
    \item Catalyst Type: None, \ce{Fe2O3}, \ce{Mg}, \ce{CuO}, \ce{NiO}, etc.
    \item Concentration: $0\%$, $0.5\%$, $1\%$, $2\%$, $3\%$, $5\%$, $10\%$ by weight
    \item Computational Tool: CEA2, PROPEP, CPROPEP, RPA, GUIPEP
\end{itemize}
\textbf{Dependent Variables:}
\begin{itemize}
    \item Specific Impulse ($I_{sp}$): Seconds
    \item Heat of Combustion ($\Delta H^{\circ}_{\text{comb}}$): kJ/kg
    \item Predictive Error: Percent error between computed and empirical values
    \item Reaction Mechanisms: From LAMMPS simulations (only qualitative/semi-quantitative inclusion)
\end{itemize}
\textbf{Controlled Variables:}
\begin{itemize}
    \item Chamber Pressure
    \item Initial Temperature
    \item Expansion Ratio
    \item Computational Settings
\end{itemize}

\subsubsection{Potential Confounding Factors}
\begin{itemize}
    \item Empirical Measurement Tools: Different laboratories use different methods for measurement (e.g. calorimetry, fixtures, and protocols). Reported $I_{sp}$ could vary due to in-person heat loss, incomplete combustion proceedings, or propellant geometry. 
    \item Manufacturing: Particle size, mixing quality, and curing conditions all hold a factor in how fast and how far equilibrium is reached, but is not mentioned since they are held constant within the scope of the study itself. 
    \item Software Implementation: Different tools might use different internal databases, numerical methods, or convergence criteria. 
\end{itemize}

\subsection{Simplified Procedures}
\begin{enumerate}
    \item Literature Review \& Data Extraction
        \begin{itemize}
            \item Extract performance metrics and formulation details from papers meeting criteria
            \item Compile data into large spreadsheet with a standardized format
            \item Target: 30-40 formulations over catalysts and concentrations
        \end{itemize}
    \item Computational Predictions
        \begin{itemize}
            \item Convert spreadsheet into input files for each of the 5 tools
            \item Record predicted $I_{sp}$ and $\Delta H^{\circ}_{\text{comb}}$ values
            \item Verify if reasonable and document any outliers
        \end{itemize}
    \item LAMMPS Molecular Dynamics Simulations
        \begin{itemize}
            \item Select 3-5 representative formulations
            \item Build models and create simulation files
            \item Analyze trajectories to identify main reaction pathways that are different from the standard uncatalyzed.
        \end{itemize}
    \item Statistical Analysis \& Framework Development
        \begin{itemize}
            \item Calculate accuracy metrics for each tools
            \item Perform statistical analysis (ANOVA, regression, correlation)
            \item Correlate MD insights with prediction accuracy from above
            \item Develop framework
        \end{itemize}
\end{enumerate}

\subsection{Analysis Criteria}

Results will be analyzed using the methods described in Rationale. Statistical software (Python with scipy/pandas) or the development of a basic LLM will be used to calculate metrics, visualizations, and performance significance tests. All data, scripts, and code will be documented and reported. 

